%% ****** Start of file template.aps ****** %
%%
%%
%%   This file is part of the APS files in the REVTeX 4 distribution.
%%   Version 4.0 of REVTeX, August 2001
%%
%%
%%   Copyright (c) 2001 The American Physical Society.
%%
%%   See the REVTeX 4 README file for restrictions and more information.
%%
%
% This is a template for producing manuscripts for use with REVTEX 4.0
% Copy this file to another name and then work on that file.
% That way, you always have this original template file to use.
%
% Group addresses by affiliation; use superscriptaddress for long
% author lists, or if there are many overlapping affiliations.
% For Phys. Rev. appearance, change preprint to twocolumn.
% Choose pra, prb, prc, prd, pre, prl, prstab, or rmp for journal
%  Add 'draft' option to mark overfull boxes with black boxes
%  Add 'showpacs' option to make PACS codes appear
%  Add 'showkeys' option to make keywords appear
\documentclass[aps,prl,preprint,groupedaddress]{revtex4}
%\documentclass[aps,prl,preprint,superscriptaddress]{revtex4}
%\documentclass[aps,prl,twocolumn,groupedaddress]{revtex4}

% You should use BibTeX and apsrev.bst for references
% Choosing a journal automatically selects the correct APS
% BibTeX style file (bst file), so only uncomment the line
% below if necessary.
%\bibliographystyle{apsrev}

\begin{document}

% Use the \preprint command to place your local institutional report
% number in the upper righthand corner of the title page in preprint mode.
% Multiple \preprint commands are allowed.
% Use the 'preprintnumbers' class option to override journal defaults
% to display numbers if necessary
%\preprint{}

%Title of paper
\title{Towards Entropic Trapping of DNA in Solid-State Nanopores}

% repeat the \author .. \affiliation  etc. as needed
% \email, \thanks, \homepage, \altaffiliation all apply to the current
% author. Explanatory text should go in the []'s, actual e-mail
% address or url should go in the {}'s for \email and \homepage.
% Please use the appropriate macro foreach each type of information

% \affiliation command applies to all authors since the last
% \affiliation command. The \affiliation command should follow the
% other information
% \affiliation can be followed by \email, \homepage, \thanks as well.
\author{Lucas Eggers}
%\email[]{Your e-mail address}
%\homepage[]{Your web page}
%\thanks{}
%\altaffiliation{}
\affiliation{Brown University}

%Collaboration name if desired (requires use of superscriptaddress
%option in \documentclass). \noaffiliation is required (may also be
%used with the \author command).
%\collaboration can be followed by \email, \homepage, \thanks as well.
%\collaboration{}
%\noaffiliation

\date{\today}

\begin{abstract}
% insert abstract here
\end{abstract}

% insert suggested PACS numbers in braces on next line
\pacs{}
% insert suggested keywords - APS authors don't need to do this
%\keywords{}

%\maketitle must follow title, authors, abstract, \pacs, and \keywords
\maketitle

% body of paper here - Use proper section commands
% References should be done using the \cite, \ref, and \label commands
\section{Introduction}

Better nanofluidic control over DNA will yield exciting leaps in technology, such as biochemical labs-on-a-chip and so-called DNA hard drives. We intend to get a few steps closer by making an isolation chamber to hold single DNA molecule in place. Such an isolation chamber would allow for the outcome of biochemical experiments to be observed on a per-molecule basis. It could also act as a storage medium for data encoded in DNA (recently estimated at a density of 2.2 petabytes per gram). We believe we can create such an isolation chamber based on entropic trapping, and combine it with a molecular detector called a nanopore.

In order to isolate a single DNA molecule in the chamber, we need a way to deliver it and to know when it enters. Nanopores are the ideal device for such purposes. A nanopore is a small hole in a membrane, be it biological or synthetic. Silicon nanopores, also known as solid-state nanopores, consist of a hole approximately 10 nanometers across in some larger piece of silicon. For contrast, double-helical DNA is approximately 2.5 nanometers across. When placed between two reservoirs of ionic solution, solid-state nanopores can detect the passage of a single DNA molecule. This passage event is also known as a translocation. When a voltage bias is placed across the two reservoirs, current forms from ions flowing through the pore. DNA, driven by electrophoretic forces on its slight negative charge, travels toward the pore, eventually getting sucked through. Its passage displaces some flowing ions, causing an observable drop in the ionic current through the pore.

In order to trap DNA in its isolation chamber, we intend to take advantage of the DNA’s configurational entropy. At equilibrium, DNA forms a random coil whose spherical shape has a size characterized by the radius of gyration (Rg). If we could get DNA in a cavity roughly the size of the radius of gyration with holes smaller than the radius of gyration on either side, we predict that the molecule will have a near-zero probability of diffusing out (Del Bonis-O’Donnel); it would have to squeeze too far out of equilibrium to exit this entropic trap.
Figure 1: A schematic of a silicon structure containing a nanopore and adjacent chamber during a DNA translocation event. The structure has 3 main layers. From bottom to top they are the nanopore, the cavity, and a large hole with diameter approximately 500 nm on average.

The device we envisioned to trap individual DNA molecules is shown in Figure 1. (Note that the molecule pictured is not trapped.) Capturing a DNA molecule in the cavity can be thought of as a competition between two forces: the force pushing the DNA through the structure and the drift of the tip of the molecule due to diffusion and electrophoretic effects. A translocation starts when the leading tip of DNA  - or someplace nearby on the polymer - enters the nanopore. A drop in current is observed. The molecule is driven through the cavity by local electric fields. The tip drifts as it crosses the cavity, possibly far enough to no longer enter the large opening when it reaches the far wall. This lateral movement is the crucial step; if the tip enters the large hole the molecule will almost certainly exit the structure, as depicted in Figure 1. The probability of exit is dependent on the cavity length and hole diameter. When the molecule finishes translocating the current will return to its original baseline value. Turning off the driving voltage as soon as possible after the translocation will allow the molecule to equilibrate inside the cavity, trapping it. Waiting too long to turn off the voltage increases the likelihood of exit by continuing to push the “skinny,” out-of-equilibrium DNA molecule toward the large exit opening. After the molecule has equilibrated, however, its expanded size would keep it in the cavity structure. 

Placing such a structure between two reservoirs of ionic solution yields the translocation dynamics described above with a twist: the structure’s asymmetry means translocations from either side of the pore are not equivalent. Although the effects of the asymmetric structure on translocation dynamics will be touched on, Karri DiPetrillo’s thesis takes a deeper look at those phenomena. The focus of this thesis will be trapping DNA molecules approaching the exposed nanopore (the bottom layer in Figure 1). Our objectives are thus twofold: to develop the hardware (pores-plus-chambers) and the software (control electronics and data analysis) to make our vision a reality. My contribution was software for analysis and recommendations for programming control electronics. In addition to creating more intricate labs-on-a-chip and possibly storing DNA hard drives, creating such entropic traps would yield deeper understanding of DNA as a polymer more generally. 
% Put \label in argument of \section for cross-referencing
%\section{\label{}}
\subsection{}
\subsubsection{}

% If in two-column mode, this environment will change to single-column
% format so that long equations can be displayed. Use
% sparingly.
%\begin{widetext}
% put long equation here
%\end{widetext}

% figures should be put into the text as floats.
% Use the graphics or graphicx packages (distributed with LaTeX2e)
% and the \includegraphics macro defined in those packages.
% See the LaTeX Graphics Companion by Michel Goosens, Sebastian Rahtz,
% and Frank Mittelbach for instance.
%
% Here is an example of the general form of a figure:
% Fill in the caption in the braces of the \caption{} command. Put the label
% that you will use with \ref{} command in the braces of the \label{} command.
% Use the figure* environment if the figure should span across the
% entire page. There is no need to do explicit centering.

% \begin{figure}
% \includegraphics{}%
% \caption{\label{}}
% \end{figure}

% Surround figure environment with turnpage environment for landscape
% figure
% \begin{turnpage}
% \begin{figure}
% \includegraphics{}%
% \caption{\label{}}
% \end{figure}
% \end{turnpage}

% tables should appear as floats within the text
%
% Here is an example of the general form of a table:
% Fill in the caption in the braces of the \caption{} command. Put the label
% that you will use with \ref{} command in the braces of the \label{} command.
% Insert the column specifiers (l, r, c, d, etc.) in the empty braces of the
% \begin{tabular}{} command.
% The ruledtabular enviroment adds doubled rules to table and sets a
% reasonable default table settings.
% Use the table* environment to get a full-width table in two-column
% Add \usepackage{longtable} and the longtable (or longtable*}
% environment for nicely formatted long tables. Or use the the [H]
% placement option to break a long table (with less control than 
% in longtable).
% \begin{table}%[H] add [H] placement to break table across pages
% \caption{\label{}}
% \begin{ruledtabular}
% \begin{tabular}{}
% Lines of table here ending with \\
% \end{tabular}
% \end{ruledtabular}
% \end{table}

% Surround table environment with turnpage environment for landscape
% table
% \begin{turnpage}
% \begin{table}
% \caption{\label{}}
% \begin{ruledtabular}
% \begin{tabular}{}
% \end{tabular}
% \end{ruledtabular}
% \end{table}
% \end{turnpage}

% Specify following sections are appendices. Use \appendix* if there
% only one appendix.
%\appendix
%\section{}

% If you have acknowledgments, this puts in the proper section head.
%\begin{acknowledgments}
% put your acknowledgments here.
%\end{acknowledgments}

% Create the reference section using BibTeX:
\bibliography{basename of .bib file}

\end{document}
%
% ****** End of file template.aps ******
